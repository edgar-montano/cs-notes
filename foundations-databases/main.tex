\documentclass[a4paper]{article}

\usepackage[english]{babel}
\usepackage[utf8]{inputenc}
\usepackage{amsmath}
\usepackage{graphicx}
\usepackage[colorinlistoftodos]{todonotes}

\title{Foundations of Databases}

\author{Edgar Montano}

\date{\today}

\begin{document}
\maketitle

\begin{abstract}
Teaches the foundation of databases.
\end{abstract}

\section{What are databases?}
\par{Solve the issue of how to store arbitrary amount of data. Databases offer an uniform way to manage and store data. Just having data does not necessarily imply you need a database. For example, we can store data into folders or possibly a spreadsheet.  Problem data that databases solve: size, ease of updating, accuracy, security, and redundancy are provided by databases opposed to traditional methods of storing data.  Databases are used to allow your data to grow (scale), they allow you to easily access your data (accessibility), they allow to keep an atomic copy, even if a lot of users are operating on it, it allows for your data to be backed up, and secure.}

\subsection{Database Management Systems (DBMS)}
\par{The term database is often misused. When programmers say their database is Oracle, MySQL, DB2, PostgreSQL, or MongoDB they are not referring to the actual database, but rather what is called as a Database Management System or DBMS for short. You use the DBMS to create and manage multiple databases. The DBMS software is the set of software that ensure a specific set of rules are applied to your database. One DBMS can manage multiple set of databases, each independent of it's own. For example, one database can manage users, another database can represent an order or customer information. DBMS are classified into different categories. One of the most common DBMS is called Relational DBMS such as Oracle, MySQL, PostgreSQL, SQLite and MS Access. }



\end{document}